%
\documentclass{article}
\usepackage{mathptmx}
\usepackage{url,latexsym,amsmath,amsthm,xspace,rotating,multirow,multicol,xspace,amssymb,paralist}
\usepackage{euscript}
\usepackage{fancybox,xcolor}
\usepackage{longtable}
\usepackage{paralist}
\usepackage[normalem]{ulem}
\usepackage[pdftex]{hyperref}

\usepackage{url}
\usepackage{latexsym}

\usepackage{times}
\usepackage{amsmath}
\usepackage{amsthm}
\usepackage{amssymb}
\usepackage{graphicx}
\usepackage{xspace}
\usepackage{tabularx}
\usepackage{multicol}
\usepackage{multirow}
%\usepackage{hyperref}
\usepackage{url}
%\usepackage{natbib}
\usepackage{wrapfig}
\usepackage{comment}
\usepackage{listings}
\usepackage{color}
\usepackage[utf8]{inputenc}
\usepackage{fancyvrb}
\usepackage{booktabs}
\usepackage{color}
\usepackage[normalem]{ulem}

\newcommand{\obs}{\text{obs}}
\newcommand{\mis}{\text{mis}}

\newcommand{\qt}[1]{\left<#1\right>}
\newcommand{\ql}[1]{\left[#1\right]}
\newcommand{\hess}{\mathbf{H}}
\newcommand{\jacob}{\mathbf{J}}
\newcommand{\hl}{HL}
\newcommand{\cost}{\mathcal{L}}
\newcommand{\lout}{\mathbf{r}}
\newcommand{\louti}{r}
\newcommand{\outi}{y}
\newcommand{\out}{\mathbf{y}}
\newcommand{\gauss}{\mathbf{G_N}}
\newcommand{\eye}{\mathbf{I}}
\newcommand{\softmax}{\phi}
\newcommand{\targ}{\mathbf{t}}
\newcommand{\metric}{\mathbf{G}}
\newcommand{\sample}{\mathbf{z}}
\newcommand{\f}{\text{f}}
%\newcommand{\log}{\text{log}}

\newcommand{\bmx}[0]{\begin{bmatrix}}
\newcommand{\emx}[0]{\end{bmatrix}}
\newcommand{\qexp}[1]{\left<#1\right>}
\newcommand{\vect}[1]{\mathbf{#1}}
\newcommand{\vects}[1]{\boldsymbol{#1}}
\newcommand{\matr}[1]{\mathbf{#1}}
\newcommand{\var}[0]{\operatorname{Var}}
\newcommand{\std}[0]{\operatorname{std}}
\newcommand{\cov}[0]{\operatorname{Cov}}
\newcommand{\diag}[0]{\operatorname{diag}}
\newcommand{\matrs}[1]{\boldsymbol{#1}}
\newcommand{\va}[0]{\vect{a}}
\newcommand{\vb}[0]{\vect{b}}
\newcommand{\vc}[0]{\vect{c}}
\newcommand{\ve}[0]{\vect{e}}

\newcommand{\vh}[0]{\vect{h}}
\newcommand{\vv}[0]{\vect{v}}
\newcommand{\vx}[0]{\vect{x}}
\newcommand{\vz}[0]{\vect{z}}
\newcommand{\vw}[0]{\vect{w}}
\newcommand{\vs}[0]{\vect{s}}
\newcommand{\vf}[0]{\vect{f}}
\newcommand{\vi}[0]{\vect{i}}
\newcommand{\vo}[0]{\vect{o}}
\newcommand{\vy}[0]{\vect{y}}
\newcommand{\vg}[0]{\vect{g}}
\newcommand{\vm}[0]{\vect{m}}
\newcommand{\vu}[0]{\vect{u}}
\newcommand{\vL}[0]{\vect{L}}
\newcommand{\vr}[0]{\vect{r}}
\newcommand{\vone}[0]{\vect{1}}
\newcommand{\mW}[0]{\matr{W}}

\newcommand{\mE}[0]{\matr{E}}
\newcommand{\mG}[0]{\matr{G}}
\newcommand{\mX}[0]{\matr{X}}
\newcommand{\mY}[0]{\matr{Y}}
\newcommand{\mQ}[0]{\matr{Q}}
\newcommand{\mU}[0]{\matr{U}}
\newcommand{\mF}[0]{\matr{F}}
\newcommand{\mV}[0]{\matr{V}}
\newcommand{\mA}{\matr{A}}
\newcommand{\mC}{\matr{C}}
\newcommand{\mD}{\matr{D}}
\newcommand{\mS}{\matr{S}}
\newcommand{\mI}{\matr{I}}
\newcommand{\td}[0]{\text{d}}
\newcommand{\TT}[0]{\vects{\theta}}
\newcommand{\vsig}[0]{\vects{\sigma}}
\newcommand{\valpha}[0]{\vects{\alpha}}
\newcommand{\vmu}[0]{\vects{\mu}}
\newcommand{\vzero}[0]{\vect{0}}
\newcommand{\tf}[0]{\text{m}}
\newcommand{\tdf}[0]{\text{dm}}
\newcommand{\grad}[0]{\nabla}
\newcommand{\alert}[1]{\textcolor{red}{#1}}
\newcommand{\N}[0]{\mathcal{N}}
\newcommand{\LL}[0]{\mathcal{L}}
\newcommand{\HH}[0]{\mathcal{H}}
\newcommand{\RR}[0]{\mathbb{R}}
\newcommand{\MM}[0]{\mathcal{M}}
\newcommand{\OO}[0]{\mathbb{O}}
\newcommand{\II}[0]{\mathbb{I}}
\newcommand{\Scal}[0]{\mathcal{S}}
\newcommand{\sigmoid}{\sigma}
\newcommand{\E}[0]{\mathbb{E}}
\newcommand{\enabla}[0]{\ensuremath{%
    \overset{\raisebox{-0.3ex}[0.5ex][0ex]{%
    \ensuremath{\scriptscriptstyle e}}}{\nabla}}}
\newcommand{\enhnabla}[0]{\nabla_{\hspace{-0.5mm}e}\,}


\newcommand{\todo}[1]{{\Large\textcolor{red}{#1}}}
\newcommand{\done}[1]{{\Large\textcolor{green}{#1}}}
\newcommand{\dd}[1]{\ensuremath{\mbox{d}#1}}

\DeclareMathOperator*{\argmax}{\arg \max}
\DeclareMathOperator*{\argmin}{\arg \min}
\newcommand{\newln}{\\&\quad\quad{}}

\newcommand{\KL}{\text{KL}}
\newcommand{\data}{\text{data}}
\newcommand{\rect}{\text{rect}}
\newcommand{\train}{\text{train}}
\newcommand{\val}{\text{val}}
\newcommand{\test}{\text{test}}
\newcommand{\Ax}{\mathcal{A}_x}
\newcommand{\Ay}{\mathcal{A}_y}
\newcommand{\ola}{\overleftarrow}
\newcommand{\ora}{\overrightarrow}
\newcommand{\ov}{\overline}
\newcommand{\ts}{\rule{0pt}{2.6ex}}       % Top strut
\newcommand{\ms}{\rule{0pt}{0ex}}         % Middle strut
\newcommand{\bs}{\rule[-1.2ex]{0pt}{0pt}} % Bottom strut
\newcommand{\specialcell}[2][c]{%
  \begin{tabular}[#1]{@{}c@{}}#2\end{tabular}}


%\usepackage{bibentry}
%\nobibliography*

\begin{document}

\title{Gradient-based Inference for Refining the Approximate Inference in
Variational Autoencoder with Discrete Latent Variables}
\author{}

\maketitle
\pagenumbering{arabic}

\section{Main}

Let us define the joint log-probability of $x$ and $h$ as
\begin{align}
    \label{eq:log_joint}
    \log p(x, h) =& \log p(h) + \log p(x|h) \\
    =& 
    \sum_{k=1}^K (h_k \log \alpha^h_k + (1 - h_k) \log (1 - \alpha^h_k))
    + \sum_{d=1}^D (x_d \log \mu^x_d + (1 - x_d) \log (1 - \mu^x_d)),
\end{align}
where $\alpha^h$ is a parameter for the prior distribution, and $\mu^x = f(h)$. Let
us use $\Psi$ to denote $\alpha^h$ as well as the parameters of $f$.

The objective in this case is to maximize the log probability of the marginal
probability of $x$:
\begin{align}
    \label{eq:approx_logp}
    \log p(x) =& \log \sum_{h} p(x, h) \nonumber \\
    =& \log \sum_h \tilde{q}(h|x) \frac{p(x, h)}{\tilde{q}(h|x)} \nonumber \\
    \geq& \sum_h \tilde{q}(h|x) \log \frac{p(x, h)}{\tilde{q}(h|x)} \nonumber \\
    =& \sum_h \tilde{q}(h|x) \log p(x,h) + \HH(\tilde{q}) \nonumber \\
    \approx& \frac{1}{N} \sum_{h^n} \log p(x, h^n) - \log
    \tilde{q}(h^n),
\end{align}
where $\HH(\tilde{q})$ is the entropy of the approximate posterior $q$.

\subsection{E Step: Approximately Inferring $p(h|x)$}

Approximately inferring $p(h|x)$ is equivalent to 
\begin{align*}
    &\argmax_{q}  \sum_h \tilde{q}(h|x) \log p(x,h) + \HH(\tilde{q}) \\
    =& \argmax_{\mu^h_1, \ldots, \mu^h_K} \sum_h \tilde{q}(h|x) \log p(x,h) + \HH(\tilde{q}),
\end{align*}
where $\mu^h_k$'s are from Eq.~\eqref{eq:log_joint}.

The issue here is that we need to {\em sample} $h$'s from $\tilde{q}(h|x)$,
which results in a high-variance, computationally-expensive estimate. Instead,
here we approximate it such that
\begin{align}
    \label{eq:grad_inf}
    \argmax_{\mu^h_1, \ldots, \mu^h_K} 
    \alpha \log p(x,\mu^h) + \HH(\tilde{q}),
\end{align}
which is equivalent to approximate $f(h)$ with $f(\mu^h)$. 
%{\color{red} This is
%equivalent to maximizing the lowebound of Eq.~\eqref{eq:approx_logp} (Need to
%check further)}.

$\alpha$ in Eq.~\eqref{eq:grad_inf} is introduced as a way to balance the
mismatch between the first approximate term and the second exact entropy term.
What we have observed is that $\log p(x, \mu^h)$ often lowerbounds the correct
term $\sum_h \log p(x, h)$, which effectively results in the underestimated
entropy. $\alpha$ can be considered as an a hyperparameter, or we can estimate
this on-the-fly during training as:
\begin{align*}
    \alpha^* = \frac{\sum_{n=1}^N \log p(x, h^n)}{\log p(x, \mu^h)},
\end{align*}
where $h^n$ is the samples generated from the approximate posterior.  We do not
directly use the Monte-Carlo estimate of $\sum_h \log p(x, h)$ to avoid an
often higher variance estimate of gradient caused by the stochastic
approximation.

Because there is a chance that this optimization may be non-convex, we
initialize the optimization from a point $\mu^{h,0}$ given by a parametric
function $q_{\TT}(h|x)$. In order to make sure that the initial point is close
to the optimum, later in the M step, we add the following auxiliary cost
function:
\begin{align}
    \label{eq:c_q}
    C_q(\TT) = \sum_{k=1}^K \mu^{h,*}_k\log q_{\TT,k}(h|x) + 
    (1 - \mu^{h,*}_k)\log (1 - q_{\TT,k}(h|x)),
\end{align}
where $\mu^{h,*}$ is the solution found by Eq.~\eqref{eq:grad_inf}. {\color{red}
    Need to check if the order is correct between $\mu^{h,*}$ and
$q_{\TT,}(h|x)$.}

\subsection{M Step: Estimating the Parameters $\Psi$ and $\TT$}

{\em Just to recap}: $\Psi$ is a set of the parameters of the generation
network, and $\TT$ is that of the recognition network.

Since the approximate inference has been computed, it is rather straightforward
to compute the gradient of Eq.~\eqref{eq:approx_logp}:
\begin{align*}
    &\nabla_{\Psi} \sum_h \tilde{q}(h|x) \log p_{\Psi}(x,h) \\
    \approx&
    \frac{1}{N} \sum_{h^n} \nabla \log p_{\Psi}(x,h^n) 
    + \log p_{\Psi}(x,h^n) \nabla \log \tilde{q}(h^n|x) 
\end{align*}
where $h^n$ is the sample from the Bernoulli distribution with the parameters
$\mu^{h,*}_1, \mu^{h,*}_2, \ldots, \mu^{h,*}_K$ obtained from the E-step.
Because $\tilde{q}$ is fixed at this point, this becomes
\begin{align}
    \label{eq:m_step}
    \frac{1}{N} \sum_{h^n} \nabla \log p_{\Psi}(x,h^n) 
\end{align}

Together with Eq.~\eqref{eq:m_step}, we minimize Eq.~\eqref{eq:c_q} together. This
results in the final cost function of
\begin{align*}
    \underbrace{\frac{1}{N} \sum_{h^n} \log p_{\Psi}(x,h^n)}_{\text{Eq.~}\eqref{eq:m_step}} + 
    \underbrace{\sum_{k=1}^K \mu^{h,*}_k\log q_{\TT,k}(h|x) + 
    (1 - \mu^{h,*}_k)\log (1 - q_{\TT,k}(h|x))}_{\text{Eq.~}\eqref{eq:c_q}}.
\end{align*}













%\bibliographystyle{abbrv}
%\bibliography{lecture_note}


\end{document}






